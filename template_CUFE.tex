\documentclass[12pt,a4paper]{article}

% Package imports
\usepackage[utf8]{inputenc}
\usepackage[english]{babel}
\usepackage{amsmath, amssymb}
\usepackage{graphicx}
\usepackage{hyperref}
\usepackage{geometry}
\usepackage{caption}
\usepackage{booktabs}

% Page layout
\geometry{margin=1in}

% Document begins
\begin{document}

% Title
\title{Your Title Here}
\author{Your Name \\ \texttt{your.email@example.com}}
\date{\today}
\maketitle

% Abstract
\begin{abstract}
This is a brief summary of your document. Write your abstract here.
\end{abstract}

% Introduction
\section{Introduction}
Introduce the topic, provide background information, and state the purpose of the document.

% Main Section
\section{Main Section Title}
Here is your main content. You can use subsections for further structure.

\subsection{Subsection Title}
Explain a specific aspect of your topic.

% Figures
\section{Figures and Tables}
Include figures and tables as needed. 

\begin{figure}[h]
    \centering
    \includegraphics[width=0.5\textwidth]{example-image} % Replace with your image file
    \caption{This is a sample figure.}
    \label{fig:sample}
\end{figure}

\begin{table}[h]
    \centering
    \begin{tabular}{|c|c|c|}
        \hline
        Column 1 & Column 2 & Column 3 \\
        \hline
        Data 1 & Data 2 & Data 3 \\
        Data 4 & Data 5 & Data 6 \\
        \hline
    \end{tabular}
    \caption{This is a sample table.}
    \label{tab:sample}
\end{table}

% Equations
\section{Equations}
Include equations like this:
\begin{equation}
    E = mc^2
\end{equation}

% Conclusion
\section{Conclusion}
Summarize the main points and provide any closing remarks.

% References
\section*{References}
\begin{thebibliography}{9}
    \bibitem{example}
    Author Name, \textit{Book Title}, Publisher, Year.
\end{thebibliography}

\end{document}
